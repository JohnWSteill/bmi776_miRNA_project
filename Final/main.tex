\documentclass{bioinfo}
\copyrightyear{2018} \pubyear{2018}

\access{Advance Access Publication Date: Day Month Year}
\appnotes{Manuscript Category}

\begin{document}
\firstpage{1}

\subtitle{Biostatistics \& Medical Informatics 776: Advanced Bioinformatics Project}

\title[short Title]{miRNA Normalization and Amplification Bias}
\author[Sample \textit{et~al}.]{John Steill\,$^{\text{\sfb 1,}*}$}
\address{$^{\text{\sf 1}}$Biostatistics \& Medical Informatics, University of Wisconsin, Madison, WI, 53792.}
\corresp{$^\ast$To whom correspondence should be addressed.}

\history{May 11, 2018}

\editor{}

\abstract{\textbf{Motivation:} Small noncoding RNAs known as microRNA (miRNA) play a crucial role in gene expression regulation. Each miRNA can target many genes, and many genes can be affected by a variety of miRNA[6]. However, there is still no consensus on how expression data should be pre-processed before downstream analysis. I will survey and evaluate existing normalization methods and attempt to characterize and correct for amplification biases. If I am successful, we will be better able to contrast the regulatory effects of miRNA profiles from different biological conditions.\\
\textbf{Results:} Text  Text Text Text Text Text Text Text Text Text  Text Text Text Text Text
Text Text Text Text Text Text Text Text Text Text Text Text Text  Text Text Text Text Text Text\\
\textbf{Availability:} Text  Text Text Text Text Text Text Text Text Text  Text Text Text Text
Text Text Text Text Text Text Text Text Text Text Text Text Text Text  Text\\
\textbf{Contact:} \href{jsteill@morgridge.org}{jsteill@morgridge.org}\\
\textbf{Supplementary information:} Supplementary data are available at \textit{Bioinformatics}
online.}

\maketitle

\section{Introduction}
\begin{itemize}
\item Motivate Problem
\item Context and Relevant work
\item State Hypothesis
\end{itemize}
Text Text Text Text Text Text  Text Text Text Text Text Text Text
Text Text  Text Text Text Text Text Text. Figure~\ref{fig:01}
shows that the above method  Text Text Text Text  Text Text Text
Text Text Text  Text Text.  \citep{Bag01} wants to know about
{\ldots}{\ldots} text follows.


%\enlargethispage{12pt}



\begin{methods}
\section{Methods}

\begin{itemize}
\item Describe Approach and Data analysis
\item Data
\item 3rd Party Algorithms
\item Developed Algorithms
\end{itemize}

\subsection{This is subheading}
\subsubsection{This is subsubheading}
\enlargethispage{6pt}

\end{methods}

\section{Results}

\section{Discussion}

\section{Figures and Tables}




\begin{table}[!t]
\processtable{This is table caption\label{Tab:01}} {\begin{tabular}{@{}llll@{}}\toprule head1 &
head2 & head3 & head4\\\midrule
row1 & row1 & row1 & row1\\
row2 & row2 & row2 & row2\\
row3 & row3 & row3 & row3\\
row4 & row4 & row4 & row4\\\botrule
\end{tabular}}{This is a footnote}
\end{table}



\begin{figure}[!tpb]%figure1
\fboxsep=0pt\colorbox{gray}{\begin{minipage}[t]{235pt} \vbox to 100pt{\vfill\hbox to
235pt{\hfill\fontsize{24pt}{24pt}\selectfont FPO\hfill}\vfill}
\end{minipage}}
%\centerline{\includegraphics{fig01.eps}}
\caption{Caption, caption.}\label{fig:01}
\end{figure}

%\begin{figure}[!tpb]%figure2
%%\centerline{\includegraphics{fig02.eps}}
%\caption{Caption, caption.}\label{fig:02}
%\end{figure}




%\bibliographystyle{natbib}
%\bibliographystyle{achemnat}
%\bibliographystyle{plainnat}
%\bibliographystyle{abbrv}
%\bibliographystyle{bioinformatics}
%
%\bibliographystyle{plain}
%
%\bibliography{Document}

\begin{equation}
\sum \text{\it x}+ \text{\it y} =\text{\it Z}\label{eq:01}\vspace*{-10pt}
\end{equation}
Figure~2\vphantom{\ref{fig:02}} shows that the above method  Text
Text Text Text  Text Text Text Text Text Text  Text Text.



\begin{thebibliography}{}

\bibitem[Bofelli {\it et~al}., 2000]{Boffelli03}
Bofelli,F., Name2, Name3 (2003) Article title, {\it Journal Name}, {\bf 199}, 133-154.

\bibitem[Bag {\it et~al}., 2001]{Bag01}
Bag,M., Name2, Name3 (2001) Article title, {\it Journal Name}, {\bf 99}, 33-54.

\bibitem[Yoo \textit{et~al}., 2003]{Yoo03}
Yoo,M.S. \textit{et~al}. (2003) Oxidative stress regulated genes
in nigral dopaminergic neurnol cell: correlation with the known
pathology in Parkinson's disease. \textit{Brain Res. Mol. Brain
Res.}, \textbf{110}(Suppl. 1), 76--84.

\bibitem[Lehmann, 1986]{Leh86}
Lehmann,E.L. (1986) Chapter title. \textit{Book Title}. Vol.~1, 2nd edn. Springer-Verlag, New York.

\bibitem[Crenshaw and Jones, 2003]{Cre03}
Crenshaw, B.,III, and Jones, W.B.,Jr (2003) The future of clinical
cancer management: one tumor, one chip. \textit{Bioinformatics},
doi:10.1093/bioinformatics/btn000.

\bibitem[Auhtor \textit{et~al}. (2000)]{Aut00}
Auhtor,A.B. \textit{et~al}. (2000) Chapter title. In Smith, A.C.
(ed.), \textit{Book Title}, 2nd edn. Publisher, Location, Vol. 1, pp.
???--???.

\bibitem[Bardet, 1920]{Bar20}
Bardet, G. (1920) Sur un syndrome d'obesite infantile avec
polydactylie et retinite pigmentaire (contribution a l'etude des
formes cliniques de l'obesite hypophysaire). PhD Thesis, name of
institution, Paris, France.

\end{thebibliography}
\end{document}
